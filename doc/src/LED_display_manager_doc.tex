\documentclass[openany,a4paper]{book}

\usepackage[utf8]{inputenc}
\usepackage[T1]{fontenc}
\usepackage[english]{babel}
\usepackage[]{color}
\usepackage{a4wide}
\usepackage{menukeys}
\usepackage[pdftex]{hyperref}

\definecolor{darkblue}{rgb}{0,0,0.5}
\definecolor{darkgreen}{rgb}{0,0.5,0}

% Url command stle (used in document for paths)
\renewcommand\UrlFont{\color{blue}\ttfamily}

% Line skip at § start
%\setlength{\parskip}{1ex plus .4ex minus .4ex}

% Configure document metainformations
\hypersetup
{
%	bookmarks=true,
	unicode=true,
	colorlinks=true,
	linkcolor=blue,
	citecolor=darkblue,
	urlcolor=darkgreen,
	pdftitle={LED display manager documentation},
	pdfauthor={Clément Foucher}
}

% Document header information
\title{LED display manager documentation}
\author
{
        Clément Foucher (\href{https://homepages.laas.fr/cfoucher}{homepage})\\
	\href{mailto:Clement.Foucher@laas.fr}{Clement.Foucher@laas.fr}\\
	\\
	LAAS--CNRS\\
	\href{https://www.laas.fr/public/en}{Laboratoire d'analyse et d'architecture des syst\`emes}\\
	\\
	\\
	\\
        \\
        \\
	\\
	\\
	\\
	Version 1.0\\
	\\
	\\
	This work is licensed under the Creative Commons\\
	Attribution-ShareAlike 4.0 International License.\\
	\\
	To view a copy of this license,\\
	visit \href{http://creativecommons.org/licenses/by-sa/4.0/}{http://creativecommons.org/licenses/by-sa/4.0/}.
	\\
	\\
	\\
}

% TODO: reorganize document: group manual installations, scripted installations.

\begin{document}

\maketitle

\tableofcontents

%%%%%%%%%%%%%%%%%%%%%%%%%%%%%%%%%%%%%%%%%%%%%%%%%%%%%%%%%%%

\newpage

\chapter*{Document revisions}

\begin{center}

\begin{tabular}{ r | c || l }
   Revision number & Date       & Changes \\ \hline \hline
   1.0             & 2017/06/07 & Initial release. \\ \hline

\end{tabular}


\end{center}


%%%%%%%%%%%%%%%%%%%%%%%%%%%%%%%%%%%%%%%%%%%%%%%%%%%%%%%%%%%

\chapter{General information}

\section{Target hardware}

The LED display manager IP is intended to command a DE-DP13212 LED display.
As the DE-DP13212 is based on a HT1632C chip, this IP could possibly be adapted to other hardware based on the same chip.

The IP uses a AXI bus, so it should run on any Xilinx device with ability to command it from a software code running on an embedded processor.
The IP can also be used without any software by using hardware direct connection or AXI interface.

Detail on how to use registers is provided in VHDL source code.

The hardware and software code provided have been tested using a ZedBoard.

\section{License}

LDM is distributed under the termes of the GNU GPL V2 license agreement. 
License agreement is available at http://www.gnu.org/licenses.

\section{Download VHDL code}

The source code of this IP can be downloaded using the following git command:
\begin{verbatim}
git clone git://git.renater.fr/leddisplaymana.git ldm 
\end{verbatim}


\chapter{Example of use on ZedBoard}

This chapter presents an example of use on ZedBoard. Vivado 2017.1 was used to generate the hardware and compile the software code.


\section{Create a Vivado IP}

In Vivado, create a new project targeting ZedBoard.

Click on \menu{Settings}, select \textit{VHDL} as target language and type \textit{led\_manager\_lib} as the default library.

Select \menu{Add sources > Add or create design sources}.
Click on \menu{Add Directories}, and add the \texttt{vhdl/} folder.

Select \menu{Tools > Create and Package New IP...}.
Select \menu{Package your current project} and choose a location to save the IP.
Answer \menu{Yes} when asked if you want to copy files.
Finally, select \menu{Review and Package} and click \menu{Package IP}.
You can close the project.

\section{Create a Vivado design}

In Vivado, create a new project targeting ZedBoard.

Select \menu{IP Catalog}, right-click on the IP list and select \menu{Add Repository...}, then choose the previously created folder containing the packaged IP.

Select \menu{Create Block Design}, then \menu{Add IP}, and select \textit{ZYNQ7 Processing System}.
Hit \menu{Run Block Automation > OK}.
Click \menu{Add IP} again, and select \textit{led\_manager\_axi\_wrapper\_v1\_0}.
Hit \menu{Run Connection Automation > OK}.

In the left panel, select the three pins \textit{data}, \textit{cs} and \textit{wr}, and press \textit{ctrl + t} to make pins external.
You can save and close the block design.

Right-click on the design on the source view and click \menu{Create HDL Wrapper... > Let Vivado manage wrapper and auto-update > OK}.

Select \menu{Add sources > Add or create constaints}, and choose the file \texttt{constraints/led\_manager\_zedboard.xdc}.

Your project is now ready to be generated.

First generate the output products: right-click on the design on the source view (it should now be listed as a child of the wrapper file), and select \menu{Generate Output Products...}.

VERY IMPORTANT: Make sure the \textit{Global} synthesis option is selected, as the design contains high impedence generation wich would be discarded if selecting other option.

Then, click on \menu{Generate Bitstream} and wait for process completion.

When done, select \menu{File > Export > Export Hardware}, check \menu{Include Bitstream} and hit \menu{OK}.

\section{Compile embedded software}

Still in Vivado project, select \menu{File > Launch SDK > OK}.

In SDK, select \menu{File > New > Application Project}, enter a project name and make sure OS is \textit{standalone} and language is C.
Click \menu{Next}, choose a \textit{Hello World} application and hit \menu{Finish}.

Right-click on the newly created project, \menu{Build Configurations > Set Active > Release}.
Delete file \texttt{src/Helloworld.c} and folder \texttt{Debug}.

Finally, right-click on \texttt{src} folder and click \menu{Import}.
Select \menu{General > File System}, and choose file \texttt{sample\_sw/led\_manager\_example.c}.

\section{Connect display to board and run}

Connect the LED display to the ZedBoard using the following connection:
\begin{itemize}
 \item cs => JA1,
 \item wr => JA2,
 \item data => JA3.
\end{itemize}

Do not forget to connect VCC and GND pins too.

Connect the ZedBoard to the computer, turn it on and select \menu{Xilinx Tools > Program FPGA}.
Right-click on the project, \menu{Run As > Run on Hardware}.

That's all, folks!


\end{document}


